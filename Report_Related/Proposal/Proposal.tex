\documentclass[conference]{IEEEtran}
\IEEEoverridecommandlockouts
% The preceding line is only needed to identify funding in the first footnote. If that is unneeded, please comment it out.
\usepackage{cite}
\usepackage{amsmath,amssymb,amsfonts}
\usepackage{algorithmic}
\usepackage{graphicx}
\usepackage{textcomp}
\usepackage{xcolor}
\def\BibTeX{{\rm B\kern-.05em{\sc i\kern-.025em b}\kern-.08em
    T\kern-.1667em\lower.7ex\hbox{E}\kern-.125emX}}
\begin{document}

\title{LEO Satellite Networks Emulator Reproduction and Study of Various LEO Satellite Routing Topologies*\\


}

\author{\IEEEauthorblockN{ Shun Zhang, Biyi He, Chengming Li}
\IEEEauthorblockA{\textit{Department of Electrical and Computer Engineering} \\
\textit{University of California San Diego}\\
San Diego \\
Emails: {xuc054, chl248}@ucsd.edu}

}

\maketitle

\section{Motivation of project}
With the realization of the Low Earth Orbit (LEO) Satellite Network, massive satellites are used to provide reliable low latency, high capacity, and high throughput. And several space companies, like SpaceX[1], are actively developing and deploying satellites into this orbit. They have developed several routing topologies and Transmission Control Protocol(TCP)/ Internet Protocol(IP)-based protocols to improve the overall network performance, i.e., latency, capacity, and throughput.	

This project aims to investigate the impact of routing topology and TCP/IP-based protocols on the performance of LEO Satellite Networks. We will delve into a range of design variables associated with LEO Satellite Networks, including satellite connection stability, latency between satellites, latency between satellites and terrestrial ground stations, as well as network capacity and throughput. The consistent connection of satellites presents challenges due to their high mobility[2]. The dynamic movement of satellites can introduce instabilities across different network layers, making it difficult to maintain a seamless connection over time. Additionally, the long-distance connections between satellites can adversely affect low-latency communication between different network layers, thus diminishing overall low-latency performance across the network system.

Design variables in the routing topology will be discussed and explored through the utilization of an emulator[3] and various routing topology papers. Various TCP/IP-based protocol designs will also be evaluated if time is allowed. Discovery and Insights will be provided and summarized.


\section{Related work}

% \subsection{Maintaining the Integrity of the Specifications}



\section{Project introduction}



\section{Timeline}


\begin{itemize}
\item Oct 16-22:
\item Oct 23-29:
\item Oct 30-Nov 5:
\item Nov 6-12:
\item Nov 13-19:
\item Nov 20-26:
\item Nov 27-Dec 3:
\item Dec 4-10:
\item Dec 11-15:
\end{itemize}

\begin{thebibliography}{00}
\bibitem{b1} Bhattacherjee, D.,  Singla, A. (2019, December). Network topology design at 27,000 km/hour. In Proceedings of the 15th International Conference on Emerging Networking Experiments And Technologies (pp. 341-354).
\bibitem{b2} Cao, X.,  Zhang, X. (2023, May). SaTCP: Link-Layer Informed TCP Adaptation for Highly Dynamic LEO Satellite Networks. In IEEE INFOCOM 2023-IEEE Conference on Computer Communications (pp. 1-10). IEEE.
\bibitem{b3}  Lai, Z., Li, H., Deng, Y., Wu, Q., Liu, J., Li, Y., ...  Wu, J. (2023). {StarryNet}: Empowering Researchers to Evaluate Futuristic Integrated Space and Terrestrial Networks. In 20th USENIX Symposium on Networked Systems Design and Implementation (NSDI 23) (pp. 1309-1324).
\bibitem{b4} 
\bibitem{b5} 
\bibitem{b6} 
\bibitem{b7} 
\end{thebibliography}


\end{document}
